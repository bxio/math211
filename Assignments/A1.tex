\documentclass[letter]{article}
\usepackage{amsmath}
\usepackage{amsfonts}
\usepackage{amssymb}
\usepackage{ifthen}
\usepackage{fancyhdr}
\usepackage[usenames,dvipsnames,svgnames,table]{xcolor}
\usepackage{tikz}

%%%
% Set up the margins to use a fairly large area of the page
%%%
\oddsidemargin=.2in
\evensidemargin=.2in
\textwidth=6in
\topmargin=0in
\textheight=9.0in
\parskip=.07in
\parindent=0in
\pagestyle{fancy}

\expandafter\def\expandafter\quote\expandafter{\quote\sf\color{DarkGreen}}

%%%
% Set up the header
%%%
\newcommand{\setheader}[6]{
  \lhead{{\sc #1}\\{\sc #2} %({\small \it \today})
  }
  \rhead{
    {\bf #3}
    \ifthenelse{\equal{#4}{}}{}{(#4)}\\
    {\bf #5}
    \ifthenelse{\equal{#6}{}}{}{(#6)}%
  }
}

%%%
% Set up some shortcut commands
%%%
\newcommand{\R}{\mathbb{R}}
\newcommand{\N}{\mathbb{N}}
\newcommand{\Z}{\mathbb{Z}}
\newcommand{\Proj}{\mathrm{proj}}
\newcommand{\Perp}{\mathrm{perp}}
\newcommand{\Span}{\mathrm{span}}
\newcommand{\Null}{\mathrm{null}}
\newcommand{\Rank}{\mathrm{rank}}
\newcommand{\mat}[1]{\begin{bmatrix}#1\end{bmatrix}}

%%%
% This is where the body of the document goes
%%%
\begin{document}
  \setheader{Math 211 (A01)}{Typed Homework 1}{Bill Xiong, Arlene Zhang}{}{}{V00737042, V00123456}

\section{Question 1}
    Let $\vec u=\mat{1\\2\\3}$, $\vec v=\mat{4\\5\\6}$, and $\vec w=\mat{7\\8\\9}$. Explain whether
          the set $A=\{\vec u,\vec v,\vec w\}$ is a basis for $\R^3$.
      Make sure to include all relevant definitions.

\subsection{Solution for Question 1}

    A \emph{basis} for a subspace $V\subset \R^3$ is a linearly independent set of vectors $A$ such that $\Span\, A=V$.

    First, let's check whether $a$ is a linearly independent set.
    A set is considered \emph{linearly independent} if the only way to express it as a linear combination of vectors is trivial. (ie, the combination where all coefficients are $0$).



    Does $\mat{0\\0\\0}=c_1\mat{1\\2\\3}+c_2\mat{4\\5\\6}+c_3\mat{7\\8\\9}$
           have a nontrivial solution?

    We can rewrite the equation above as:
        \begin{align*}
          0&=c_1+4c_2+7c_3\\
          0&=2c_1+5c_2+8c_3\\
          0&=3c_1+6c_2+9c_3.
        \end{align*}
        \\
    FIXME: SHOW MORE WORK HERE\\
  By solving this system, we can conclude that any $c$'s satisfying the condition $c_2=-2c_3=-2c_1$ will lead to a valid solution.
  If we arbitrarily chose the value of $1$ for $c_1$, we can plug in and find that $c_2 = -2$ and $c_3 = 1$ give a solution to the above system.
  Since $c_1$, $c_2$, and $c_3 \neq 0$, $\vec 0=\vec u-2\vec v+\vec w$ is a solution.

Since $\vec 0$ can be written as a non-trivial linear combination of vectors in $A$, $A$ is a linearly dependent set, and therefore cannot be a basis for $R^3$.

\section{Question 2}

Fix $\vec u,\vec v\in \R^n$.  Show that $\Span(\Span\{\vec u,\vec v\})=\Span\{\vec u,\vec v\}$.
      Make sure to include all relevant definitions.

\subsection{Solution for Question 2}

    The \emph{span} of a set of vectors $X$ is defined as set of all linear combinations of vectors in $X$.

    We want to show that for any two vectors $\vec u,\vec v\in \R^n$,
        $\Span(\Span\{\vec u,\vec v\})=\Span\{\vec u,\vec v\}$. To do this, we show that $\Span(\Span\{\vec u,\vec v\})$ and $\Span\{\vec u,\vec v\}$ are subsets of each other.

  Let $\vec u,\vec v$ be some arbitrary vectors in $\R^n$ and $A=\Span\{\vec u,\vec v\}$.

  $A=\{\vec x\in\R^n:\vec x=m\vec u+n\vec v$ for some $m,n\in \R\}$.

    If we also define $\vec w$ as some arbitrary vector in $\Span\,A$, then by definition
    $\vec w=\sum_{i=1}^n k_i\vec s_i$. for some $s_i \in \vec A$ and some value $k \in \R$.

By the definition of $A$, we can expand the above into:

\[
    \vec w =\sum_{i=1}^n k_i (m_i\vec u+n_i\vec v)=
          (\sum_{i=1}^n k_i m_i)\vec u+
          (\sum_{i=1}^n k_i n_i)\vec v
\]

 Expanding based on the definition
        of $A$, we see
        \[
          \vec w =\sum_{i=1}^n \gamma_i (\alpha_i\vec u+\beta_i\vec v)=
          \left(\sum_{i=1}^n \gamma_i \alpha_i\right)\vec u+
          \left(\sum_{i=1}^n \gamma_i \beta_i\right)\vec v
        \]
        for some $m_i,n_i\in \R$.

        However, the right hand sides expresses
        $\vec w$ as a linear combination of $\vec u$ and $\vec v$.  Hence, $\vec w\in A$.
        This shows $\Span\, A\subseteq A$.  Since $A$ is obviously a subset of  $\Span\,A$, $A=\Span\,A$.


\section{Question 3}

The worksheets define $\Proj_{\vec v}\vec u$ as the vector in the direction $\vec v$ such that
      $\vec u-\Proj_{\vec v}\vec u$ is orthogonal to $\vec v$.  Call this definition (a).  Your textbook
      defined $\Proj_{\vec v}\vec u$ as the vector $\frac{\vec u\cdot \vec v}{\vec v\cdot\vec v}\vec v$.
      Call this definition (b).  Show that definitions (a) and (b) are equivalent by showing
      that that the vector arising from definition (b) must be the same as the vector
      arising from definition (a).
      In your answer, elaborate on definition (a) by including
      the definition of \emph{vector in the direction of $\vec v$} and \emph{orthogonal}.

\subsection{Solution for Question 3}
FIXME:SOLUTION For 3 goes here\\

\section{Addendum}
  This is the first time one of us is submitting an assignment in \LaTeX{}. Please be gentle in grading our assignment and excuse the obvious poor formatting.
\end{document}
