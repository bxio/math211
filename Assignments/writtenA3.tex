\documentclass[letter]{article}
\usepackage{amsmath}
\usepackage{amsfonts}
\usepackage{amssymb}
\usepackage{ifthen}
\usepackage{fancyhdr}
\usepackage[usenames,dvipsnames,svgnames,table]{xcolor}
\usepackage{tikz}

%%%
% Set up the margins to use a fairly large area of the page
%%%
\oddsidemargin=.2in
\evensidemargin=.2in
\textwidth=6in
\topmargin=0in
\textheight=9.0in
\parskip=.07in
\parindent=0in
\pagestyle{fancy}

\expandafter\def\expandafter\quote\expandafter{\quote\sf\color{DarkGreen}}

%%%
% Set up the header
%%%
\newcommand{\setheader}[6]{
  \lhead{{\sc #1}\\{\sc #2} %({\small \it \today})
  }
  \rhead{
    {\bf #3}
    \ifthenelse{\equal{#4}{}}{}{(#4)}\\
    {\bf #5}
    \ifthenelse{\equal{#6}{}}{}{(#6)}%
  }
}

%%%
% Set up some shortcut commands
%%%
\newcommand{\R}{\mathbb{R}}
\newcommand{\N}{\mathbb{N}}
\newcommand{\Z}{\mathbb{Z}}
\newcommand{\proj}{\mathrm{proj}}
\newcommand{\Span}{\mathrm{span}}
\newcommand{\Null}{\mathrm{null}}
\newcommand{\Rank}{\mathrm{rank}}
\newcommand{\mat}[1]{\begin{bmatrix}#1\end{bmatrix}}

%%%
% This is where the body of the document goes
%%%
\begin{document}
  \setheader{Math 211 (A01)}{Written Homework 3}{Due Friday, February 27}{}{Bill Xiong}{V00737042}


\section {Question 1}
For each of the following statements, produce a counterexample to show that the statement is {\bf false}.
      \begin{enumerate}
        \item If $A$ and $B$ are square matrices, $AB=BA$.
        \item If $AB=\mat{1&1\\1&1}$, then $A$ and $B$ are $2\times 2$ matrices.
        \item If $AB=I$ then $BA=I$.
        \item If $A^2=0$, then $A=0$.
      \end{enumerate}

\subsection{1A}
Let $A = \mat{1&2\\3&4}$, $B = \mat{1&0\\2&1}$. $AB = \mat{5&2\\11&4}$ but $BA = \mat{1&2\\5&8}$.
\subsection{1B}
Let $A = \mat{1&0&0\\1&0&0}$ and $B = \mat{1&1\\0&0\\0&0}$. $AB = \mat{1&1\\1&1}$ but $A$ and $B$ are not square.
\subsection{1C}
Let $A = \mat{1&1&0\\.1&0&1}$ and $B = \mat{0&0\\1&0\\0&1}$. $AB = \mat{1&0\\0&1} = I$ but $BA = \mat{0&0&0\\1&1&0\\1&0&1} \neq I$
\subsection{1D}
Let $A = \mat{0&1\\0&0}$. Then $A^2 = 0$ but $A\neq0$.
\section{Question 2}
Let $R=\mat{1&2&3\\4&5&6\\7&8&9}$.
      \begin{enumerate}
        \item Find all solutions to the matrix equation $R\mat{x_1\\x_2\\x_3}=\mat{2\\5\\8}$.
        \item Prove that the set $X=\{\vec x\in\R^3:R\vec x=\vec 0\}$ is a subspace.
      \end{enumerate}
\section{Question 3}
Suppose $E$ is a $4\times 3$ matrix with columns $\vec c_1,\vec c_2,\vec c_3$ and rows
      $\vec r_1,\vec r_2,\vec r_3,\vec r_4$.  Let $\vec v=\mat{2\\-1\\1}$.
      \begin{enumerate}
        \item Express $E\vec v$ as a linear combination of $\vec c_1,\vec c_2,\vec c_3$.
        \item Supposing $\vec r_1\cdot \vec v=1$, $\vec r_2\cdot \vec v=6$, $(\vec r_3+\vec r_4)\cdot \vec v=2$,
          and $(\vec r_3-\vec r_4)\cdot \vec v=-2$, compute $E\vec v$.
      \end{enumerate}
\section{Question 4}
Suppose that $\vec u$, $\vec v$, and $\vec w$ are vectors in $\R^2$ that are related
      by the following diagram.
    \begin{center}
      \begin{tikzpicture}[scale=.5]
        \draw[-] (-6, 0) -- node [below,
        very near end] {} (6, 0);
        \draw[-] (0, -2) -- node [right,
        very near start] {} (0, 5);
        \draw[->] (0, 0) -- node [below,
        very near end] {$4\vec{u}$}
        (3, -1);
        \draw[->] (0, 0) -- node [left,
        near end] {$2\vec{v}$} (2, 3);
        \draw[->] (0, 0) -- node [left,
        very near end] {$3\vec{w}$}
        (-1, 4);
        \draw[dotted] (3, -1) -- node
        [right, very near end] {} (2, 3);
        \draw[dotted] (-1,4) -- node
        [below, very near end] {} (2, 3);
      \end{tikzpicture}
    \end{center}
    Let $A=[\vec u|\vec v|\vec w]$ be the matrix with columns $\vec u$, $\vec v$, and $\vec w$.
    \begin{enumerate}
      \item What is the rank of $A$?
      \item Find all solutions to the equation $A\vec x=\vec 0$.
      \item Find a basis for the subspace $V=\{\vec x\in\R^3: A\vec x=\vec 0\}$.
    \end{enumerate}

\section{Addendum}

\end{document}
